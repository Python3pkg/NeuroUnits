
\documentclass{article}


\usepackage{verbatim} 
\usepackage{listings} 
\usepackage[usenames,dvipsnames]{color}
\usepackage[margin=2cm]{geometry}
%\usepackage{alltt}
\usepackage{syntax}
\usepackage{amsmath}


\title{Proposal for a  grammar for units and expressions involving units in computational neuroscience}
\author{Michael Hull}


\lstset{% general command to set parameter(s)
basicstyle=\footnotesize, % print whole listing small
keywordstyle=\color{blue}, %\bfseries\underbar,
identifierstyle=, % nothing happens
commentstyle=\color{OliveGreen}, % white comments
stringstyle=\ttfamily, % typewriter type for strings
showstringspaces=false, % no special string spaces
language=python,
frame=lines,
float=tbfh}

\begin{document}
\maketitle

\tableofcontents

\section{Motivation}

Computational neuroscience is a field with a large 
array of tools aimed at specific tasks. Recently, 
there has been a focus on tool interoperability. 
Unfortunately, many different tools use different 
notations for unit handling.\\


In general, the parsing of expressions involving units and constants is tricky. For example the expression
\verb| val = 1 K / F | could mean \emph{'one-kelvin-per-Farad'} or it could be \emph{'one-kelvin-divided by the gas constant'}. \\

We propose a simple, unambiguous, ascii-string based format for denoting quantities with units, and simple expressions involving these quantities, which can be implemented with standard parsing tools. BNF grammars are provided.  Particular emphasises are human readbility, extensibility and inter-operability with existing conventions.  For adoption by the community, implementation should be as simple as possible, tools developers should only need to implement features that they need.  



\newpage
\section{Review  of unit definition in existing tools}

\subsection*{Neuron}
\verb|NEURON| separates the simulation process into two parts; \verb|.hoc| and \verb|.mod| files. The units that a particular parameter will be specified in are specified in \verb|.mod| files. NEURON comes with a database of units and constants, derived from the GNU units packages, found in \verb|nrnunits.lib|. New units and constants can also be defined in terms of existing expressions within the \verb|UNITS| block declaration. For example.

\begin{verbatim}
UNITS {
    : Define units:
   (uF)    =  (microfarad)
   (Mohms) =  (megohms)
   (V)     =  (volt)
   (molar) =  (/liter)
   (mM)    =  (millimolar)
    
    : Define constants:
    F = (faraday) (coulomb)
    PI = (pi) (1)
    e  = (e) (coulomb)
    R  = (k-mole) (joule/degC)
    C = (c) (cm/sec)
}
\end{verbatim}



Although units can be supplied, there is no automatic conversion between them; the specification of units allows the units to be validated with the commandline tool \verb|modlunits|. To allow this, \verb|NEURON| requires a special syntax when composing expressions, so that \verb|modlunits| can disambiguate between what is a value and what is a conversion factor. For example, it is invalid to write:
\begin{verbatim}
ASSIGNED{
    i (milliamp)
    v (volt) 
    r (ohm)
}
EQUATION {
    : INVALID - no automatic conversion between volts <--> milliamp * ohm (=millivolt)
    v = i * r 
    : INVALID - numerically it is correct, but modlunits can't detect the conversion factor
    v = 0.001 * i * r

    : VALID - conversion factor is in parathesis
    v = (.001) * i * r
}
\end{verbatim}

http://cns.iaf.cnrs-gif.fr/files/units.html



\subsection*{Genesis}
TOCHECK: It appears to me that units in genesis are all specified using SI base units. But  I am not a genesis user - is this true??


\subsection*{SBML}
SBML is an XML format for specifying models in systems biology. New unit definitions can built from builtin types (or existing definitions, I think!?).
For example

\begin{lstlisting}

<unitDefinition id="litre_per_mole_per_second">
                <listOfUnits>
                    <unit kind="mole"   exponent="-1"/>
                    <unit kind="litre"  exponent="1"/>
                    <unit kind="second" exponent="-1"/>
                </listOfUnits>
</unitDefinition>
\end{lstlisting}



\subsection*{BRIAN}
BRIAN allows the specification of units in python code, by writing for example:

\begin{lstlisting}
from brian import *
tau = 20 * msecond        # membrane time constant
Vt = -50 * mvolt          # spike threshold
Vr = -60 * mvolt          # reset value
\end{lstlisting}

and allows validation of expected unit types by writing:
\begin{lstlisting}
model='dV/dt = -(V-El)/tau : volt'
\end{lstlisting}


\subsection*{GNU units}
GNU units is an interactive commandline program that allows the user to convert from one unit to another.
For example
\begin{verbatim}
You have: 20 mph
You want: sec/mile
    reciprocal conversion
    1 / 20 mph = 180 sec/mile
    1 / 20 mph = (1 / 0.0055555556) sec/mile
\end{verbatim}
GNU units has a large library of built in units, multipliers and constants.



\newpage
\section{Proposed requirements} 
\subsection{Use cases}

\begin{description}
\item[Simulation configuation:] \hfill \\
Most simulators either allow direct access to thier API through a programming language, or read simulation configurations in from files. The a flexible specification for units would be an inline ascii-string.  For example, to specify a leak channel, we would like to be able to write:
\begin{lstlisting}
create_leak_channel(reversal_potential="-70 mV", conductance="3 pS/cm2")
\end{lstlisting}
Often, data is integrated from biological data directly, it would be easier to
check the values used in the simulation if we are able to write something like:
\begin{lstlisting}
create_leak_channel(reversal_potential="-70 mV", conductance="(1/300 MOhm)/(500um2)")
\end{lstlisting}


\item[Loading \& storing of data \& results:]  \hfill \\


Often, data is stored in an intermediate format between tools, either in text files (eg. .csv files), or in binary files, (.mat, .hdf) files. These formats generally allow the storage of associated metadata with arrays or data in the form of strings. To aid interoperability between tools, and not require the user to manually keep track of whether a current density measurement is stored in for example \verb|pA/um2| or \verb|mA/cm2|, a unit string would allow us to write, for example:

\begin{verbatim}
# My CSV FILE
# COLUMN0: {'label':'t', 'unit':'ms' }
# COLUMN1: {'label':'membrane_voltage', 'unit':'mV' }
# COLUMN2: {'label':'membrane_current', 'unit':'pS/cm2' }
0.000, -75.000  2.300
0.010, -75.000  2.300
0.020, -75.000  2.300
0.030, -75.000  2.300
\end{verbatim}


\item[In specification of functions and formulae] \hfill \\
Specifying units directly in equations would make explicit certain conventions.
For example, in hodgekin huxley formulation, we often see equations of the form:
$$
m_\alpha = \frac{A + BV}{C + exp( \frac{D+V}{E}) }
$$

Since the dimensions of $m_\alpha$ are 'per second', so the dimensions for the parameters are:
\begin{itemize}
\item[A:]  'per second'
\item[B:]  'per (volt second) 
\item[C:]  (None)
\item[D:]  'volt'
\item[E:]  'volt'
\end{itemize}

Often, for clarity, when written, the units are dropped, and in \verb|NEURON|, the code is written as (surrounding code removed for clarity):

\begin{lstlisting}
ASSIGNED {
    v (mV)
    : code removed...
}

PROCEDURE rates(v(mV)) {  :Computes rate and other constants at current v.
UNITSOFF
    : code removed...
    alpha = .07 * exp(-(v+65)/20)
    : code removed...
}
\end{lstlisting}
The code for units checking in this case becomes unwieldly and has been removed by the UNITSOFF directive. It would be good if our syntax did not encourage this; either by forcing users to be explcit about how to convert quantities with units to quantities without units, or by making the specification and use of the units for \verb@of A,B,C,D & E@ simpler.







\end{description}



\subsection{Constraints}
\begin{description}
\item[Scope] The aim is not to solve the general specification of units, but provide neuroscience specific units and commonly used constants. We don't want to have to redefine 'volt' or 'siemen' each time from base quanties, there should be a builtin set of standard units. We also want to be able to define our own dimensions and constants.
\item[Readability] Its very important that the syntax is human readable. Excessive use of 'delimiting characters' should be avoided.
\item[Implementation] Easy to do for adoption. Needs to be simple to implement parsers. Tricky cases arise because of ambiguous cases: 'g'(gravity, gram), 'e' (2.71, electric charge), spellings of 'meter', 'F' (farad/gas constants).
\end{description}

\subsection{Example Formula}
\begin{description}
\item NMDA? Cellular concentrations.
\end{description}

\newpage
\section{Proposal}

\subsection{Overview}
\subsubsection*{Units}
Units can be specified as for example:
\begin{verbatim}
m
m/s 
m/s2 
kg m2 s-1
milliamp/centimeter2
m/(second volt) 
pS/(m2 /s3)
\end{verbatim}

Brackets are valid, there is NO * operator; WHITESPACE is multiplication and has higher preceedence that division. For example
\verb|kg s2 / F meter| is interpreted as \verb| (kg s2) / (F meter)|

To maintain convention with left-right  application of '/' as in languages like C and python, the interpretation of  \verb|"kg / m / F / s"| would be the same as \verb|"kg / m  F  s"|. Having redundant '/' is confusing, so there can not be more than one division sign without intervening brackets.
\verb| ((kg/mol)/m2)/(pS/(um/s))| is valid, but \verb|kg/s/s| is not and can be written \verb|m/(s s)|, or \verb|(m/s)/s| (or \verb|m/s2|).


\subsubsection*{Unit Quantities}

Unit quanties can be defined as an integer or a float followed by a unit, separated by an optional whitespace. 
Units quantities can also be defined in terms of externally defined form, using the syntax
\verb| 2.3 {~myspecialunit}|.  See the lexing section for details about floating point and externally defined string specification.



There is the potential for ambiguity with multiplication and 
Definitions of units, expressions, unitdefinitons
Examples

\subsubsection*{Unit Expressions}
Unit quantities can be added, subtracted, multiplied and divided, and used in expressions in brackets. The preceedence is the same as C.
Constants can be used, and are enclosed in \verb|{}|.
For example:
\begin{verbatim}
(1/500MOhm) / 590um2
NERST EXAMPLE
\end{verbatim}

An expression can be suffixed with a colon, to confirm the resulting unit of the expression. (The multipliers need not be the same, but the expression must be convertable). For example:
\begin{verbatim}
mS/cm2 * mV : pS/um2 
\end{verbatim}


\subsection{Function Calls}



\subsection{Syntax Limitations}
Use of numbers in variable names.

\subsection*{Note about Preceedence}
Multiplication can be funny, and division.

\subsection*{Funky examples with whitespace}
Multiplication can be funny, and division.


\newpage
\subsection{Proposed Builtins}




\subsubsection{Base units}

\begin{center}
    \begin{tabular}{ | l | l |  l | p{5cm} |}
    \hline
    long-form & short-form & value  \\ \hline
    meter & m & - \\ \hline
    gram & g & - \\ \hline
    second & s & - \\ \hline
    ampere & A & - \\ \hline
    kelvin & K & - \\ \hline
    mole & mol & - \\ \hline
    candela & cd & - \\ \hline
    \end{tabular}
\end{center}

\subsubsection{Multipliers}

\begin{center}
    \begin{tabular}{ | l | l |  l | p{5cm} |}
    \hline
    long-form & short-form & value  \\ \hline
    giga & G & $10^9$ \\ \hline
    mega & M & $10^6$ \\ \hline
    kilo & k & $10^3$ \\ \hline
    centi & c & $10^{-2}$ \\ \hline
    milli & m & $10^{-3}$ \\ \hline
    micro & u & $10^{-6}$ \\ \hline
    nano & n & $10^{-9}$ \\ \hline
    pico & p & $10^{-12}$ \\ \hline
    \end{tabular}
\end{center}




\subsubsection{Neuroscience Specific Constants}
\begin{center}
    \begin{tabular}{ | l | l |  l | p{5cm} |}
    \hline
    Symbol & Description & value  \\ \hline
    Na & Avogadro's Constant & - \\ \hline
    \end{tabular}
\end{center}



\subsubsection{Neuroscience Specific Units}

\begin{center}
    \begin{tabular}{ | l | l |  l | p{5cm} |}
    \hline
    long-form & short-form & base-units  \\ \hline
    volt & V & - \\ \hline
    siemen & S & - \\ \hline
    farad & F & - \\ \hline
    coulomb & C & - \\ \hline
    hertz & Hz & - \\ \hline
    ohm & Ohm & - \\ \hline
    watt & W & - \\ \hline
    joule & J & - \\ \hline
    newton & N & - \\ \hline
    liter & l & - \\ \hline
    BLAH & BLAH & - \\ \hline
    \end{tabular}
\end{center}


\subsubsection{Mathematical Constants}
\begin{center}
    \begin{tabular}{ | l | l |  l | p{5cm} |}
    \hline
    Symbol & Value & Units & Description  \\ \hline
    pi & 3.141 & - & Radians in a circle \\ \hline
    e\_euler & - & - & - \\ \hline
    \end{tabular}
\end{center}

\subsubsection{Mathematical Functions}
\begin{center}
    \begin{tabular}{ | l |  p{5cm} |}
    \hline
    Signature &  Description  \\ \hline
    log2 & {...} \\ \hline
    log10 & {...} \\ \hline
    \end{tabular}
\end{center}

\newpage
\subsection{Parsing Algorithm}

Parsing units is not a trivial process. Unfortunately, whitespace plays an important role, and the overlap of the symbol 'm' in the multipliers and base units complicates the grammar. However, it is still quite simple using 2 stage parsing stages, one to handle epxressions, and the other to parse individual unit terms:


\subsubsection{Preprocessing}

White space is very important in interpreting these equations. We apply the following regular expressions to simplify the parsing grammar.

\begin{description}
\item[Strip unnessesary whitespace] \hfill \\
Remove all the whitespace surrounding the symbols: \verb@ + ( ) / * :@  \\
\verb@"[ \t]* ([()/*:+]) [ \t]*"@ $\rightarrow$ \verb@"\1"@

\item[Substitute Division Sign] \hfill \\
Since \verb@/@ plays two roles, either division of unit-terms (eg mV/pS) or division of
quantity terms (1mV/1pS), we remap the division of unit terms to '//' \\
\verb@"/(?= [(]* (?:[a-zA-Z]) )"@ $\rightarrow$ \verb@"//"@ \\
\verb@"/(?= [(]* (?:{~) )"@ $\rightarrow$ \verb@"//"@

\item[Substitute Subtraction Signs] \hfill \\
Similarly, \verb@-@ can play three roles, either in an unit term (eg kg-2 m), as a subtraction
operator (1mV - 1V) or as a numerical modifier (1mV + -1mV) \\
\verb@"[ ]* [-](?=[^0-9]) [ ]*"@ $\rightarrow$ \verb@"--"@

\end{description}


\subsubsection{Expression Lexing Tokens}

{\scriptsize
\begin{equation}
\begin{aligned}
\verb!NO_UNITS!       & \verb! := !       \verb!NO_UNIT! \\
\verb!INTEGER!        & \verb! := !       \verb![+-]?[0-9]+! \\
\verb!FLOAT!          & \verb! := !       \verb![-]?[0-9]+\.[0-9]*([eE][+-]?[0-9]+)?! \\
\verb!ALPHATOKEN!     & \verb! := !       \verb![a-zA-Z_]+! \\
\verb!SLASHSLASH!     & \verb! := !       \verb!//!\\
\verb!SLASH!          & \verb! := !       \verb!/!\\
\verb!WHITESPACE!     & \verb! := !       \verb![ \t]+"""!\\
\verb!LBRACKET!       & \verb! := !       \verb!\(!\\
\verb!RBRACKET!       & \verb! := !       \verb!\)!\\
\verb!LCURLYBRACKET!  & \verb! := !       \verb!\{!\\
\verb!RCURLYBRACKET!  & \verb! := !       \verb!\}!\\
\verb!EXCLAIMATION!   & \verb! := !       \verb@!@\\
\verb!TILDE!          & \verb! := !       \verb!~!\\
\verb!TIMES!          & \verb! := !       \verb!\*!\\
\verb!PLUS!           & \verb! := !       \verb!\+!\\
\verb!COMMA!          & \verb! := !       \verb!,!\\
\verb!COLON!          & \verb! := !       \verb!:!\\
\verb!EQUALS!         & \verb! := !       \verb!=!
\end{aligned}
\end{equation}
}


\setlength{\grammarparsep}{20pt plus 1pt minus 1pt} % increase separation between rules
\setlength{\grammarparsep}{15pt} % increase separation between rules
\setlength{\grammarindent}{12em} % increase separation between LHS/RHS 

\newpage
\subsubsection{Expression Grammar BNF}

This is a high level grammar, which parses expressions. The term in "unit\_term\_unpowered" 
need to be parsed onto the next parser, since the parsing of individual unit terms requires a 
separate lookup term to resolve the ambiguity between 'm' in milli or meter.

{\scriptsize
\begin{grammar}

<unit\_term\_unpowered> ::= ALPHATOKEN

<unit\_term> ::= <unit\_term\_unpowered>
\alt <unit\_term\_unpowered> INTEGER
\alt LCURLYBRACKET TILDE ALPHTOKEN RCURLYBRACKET

<unit\_term\_grp> ::= <unit\_term>
\alt <unit\_term\_grp> WHITESPACE <unit\_term>

<parameterised\_unit\_term> ::= LBRACKET <unit\_term\_grp> RBRACKET
\alt LBRACKET <unit\_term\_grp> SLASHSLASH <unit\_term\_grp> RBRACKET


<unit\_expr> ::= <unit\_term\_grp>
\alt <unit\_term\_grp> SLASHSLASH <unit\_term\_grp>
\alt <parameterised\_unit\_term> SLASHSLASH <parameterised\_unit\_term>
\alt <unit\_term\_grp> SLASHSLASH <parameterised\_unit\_term>
\alt <parameterised\_unit\_term> SLASHSLASH <unit\_term\_grp>
\alt <parameterised\_unit\_term>


<quantity> ::= magnitude
\alt <magnitude> <unit\_expr>
\alt <magnitude> WHITESPACE <unit\_expr>
\alt LCURLYBRACKET <constant> RCURLYBRACKET 
\alt NO\_UNIT LCURLYBRACKET quantity\_expr RCURLYBRACKET 

<constant> ::= ALPHATOKEN 

<magnitude> ::= FLOAT 
\alt INTEGER

<quantity\_expr> : <quantity\_expr> PLUS <quantity\_term>
\alt : <quantity\_expr> MINUSMINUS <quantity\_term>
\alt : <quantity\_term>

<quantity\_term> :  <quantity\_term> TIMES <quantity\_factor>
\alt : <quantity\_term> SLASH <quantity\_factor>
\alt : <quantity\_factor>

<quantity\_factor> : <quantity> 
\alt LBRACKET <quantity\_expr> RBRACKET 

<func\_call\_param> : ALPHATOKEN EQUALS quantity\_expr>
<func\_call\_params> : quantity\_expr>
\alt <func\_call\_param> 
\alt <func\_call\_params> COMMA <func\_call\_param>

<function\_def> : ALPHATOKEN LBRACKET <function\_def\_params> RBRACKET EQUALS <quantity\_expr>

<function\_def\_params> : ALPHATOKEN
\alt ALPHATOKEN COLON <unit\_expr>
\alt <function\_params> COMMA ALPHATOKEN

\end{grammar}
}


\newpage
\subsubsection{Token-Parsing Grammar BNF}

The "unit\_term\_unpowered" also requires an additional lookup step. If the 
unit is 'm,cm,um,....', then the unit should be resolved as the relevant unit directly.

{\scriptsize
\begin{grammar}
<unit\_term\_unpowered> ::=  <short\_basic\_unit> 
\alt <long\_basic\_unit>
\alt <SPECIAL-SEE TEXT>

<unit\_term\_unpowered> ::=  <long\_basic\_multiplier> <long\_basic\_unit>
\alt <short\_basic\_multiplier> <short\_basic\_unit> 

<long\_basic\_unit> ::=  LONG\_VOLT 
\alt LONG\_AMP 
\alt LONG\_SIEMEN
\alt LONG\_OHM
\alt LONG\_COULOMB
\alt LONG\_FARAD
\alt LONG\_METER
\alt LONG\_GRAM
\alt LONG\_SECOND
\alt LONG\_KELVIN
\alt LONG\_MOLE
\alt LONG\_CANDELA

<short\_basic\_unit> ::= SHORT\_VOLT 
\alt SHORT\_AMP 
\alt SHORT\_SIEMEN
\alt SHORT\_OHM
\alt SHORT\_COULOMB
\alt SHORT\_FARAD
\alt SHORT\_METER
\alt SHORT\_GRAM
\alt SHORT\_SECOND
\alt SHORT\_KELVIN
\alt SHORT\_MOLE
\alt SHORT\_CANDELA

<long\_basic\_multiplier> ::=  LONG\_GIGA 
\alt LONG\_MEGA 
\alt LONG\_KILO
\alt LONG\_CENTI
\alt LONG\_MILLI
\alt LONG\_MICRO
\alt LONG\_NANO
\alt LONG\_PICO

<short\_basic\_multiplier> ::=   SHORT\_GIGA 
\alt SHORT\_MEGA                           
\alt SHORT\_KILO                           
\alt SHORT\_CENTI                          
\alt SHORT\_MILLI                          
\alt SHORT\_MICRO                          
\alt SHORT\_NANO                           
\alt SHORT\_PICO                           
                              
\end{grammar}
}


\subsection{Future Extensibility}
Function call defintoins
Derivatives

% {xyz}  Builtin Constant 
% {$xyz} Parameter/State
% {@xyz} Parameter [for function call]
% {~xyz} units








\newpage
\section{Examples}
\subsection{Valid Units}
\verbatiminput{../../src/testing/validunits.txt}

\subsection{Valid Expressions}
\verbatiminput{../../src/testing/validexpressions.txt}


\subsection{Invalid Expressions}



\end{document}
