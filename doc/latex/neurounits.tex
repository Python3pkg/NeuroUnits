
\documentclass{article}


\usepackage{verbatim} 
\usepackage{listings} 


\title{A grammar for units in computational neuroscience}
\begin{document}

\maketitle

\section*{Motivation}

Computational neuroscience is a field with a large 
array of tools aimed at specific tasks. Recently, 
there has been a focus on tool interoperability. 
Unfortunately, many different tools use different 
notations for unit handling. 

A simple, unambiguous, ascii-string based format for denoting quantities with units, and 
simple expressions involving these quantities, which can be implemented with standard parsing tools. 


\section*{Review  of unit definition in existing tools}

\subsection*{Neuron}
NEURON separates the simulation process into two parts; .hoc and .mod files. The units that a particular parameter will be specified in are specified in .mod files. NEURON comes with a database of units and constants, derived from the GNU units packages, found in 'nrnunits.lib'. New units and constants can also be defined in terms of old units. E.G.
\subsection*{Genesis}
\subsection*{SBML}
\subsection*{BRIAN}



\section*{Use Cases \& Constraints}
Use-cases:

\begin{itemize}
\item[Simulator Configuation]   
Since most programming languages support character string, we would like to be able to write
\begin{lstlisting}
\verb|create_leak_channel(reversal_potential="-70 mV", conductance="3 pS/cm2")|
\end{lstlisting}
, or, ideally, to make comparisons with physiology easier, \verb|create_leak_channel(reversal_potential="-70 mV", conductance=" (1/300 MOhm)/(500um2)")|, to specify the a leak channel for a biophysical neuron.

\item[Loading \& Storing of Data \& Results]


\end{itemize}


Constraints:
\begin{itemize}
\item[Scope] Neuroscience specific 
\item[Implementation] Neuroscience specific 
\item[Implementation] Simple to implement
\end{itemize}

Example Formula
\begin{itemize}
\item NMDA? Cellular concentrations.
\end{itemize}



\section{Grammar}









\end{document}
